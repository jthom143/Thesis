\begin{abstract}
\addcontentsline{toc}{chapter}{Abstract}

Natural variability of the atmosphere and ocean are important processes to
understand and quantify in order to accurately detect and predict anthropogenic
climate change. In order to investigate and quantify natural variability
multiple global climate models (GCMs) are used along with observational data
to investigate the multi-decadal natural variability of three processes:

First, the natural variability of the Southern Annular Mode (SAM) and Southern
Hemisphere westerly jet strength and position are quantified using 14 different
GCMs. The magnitude of the natural variability of these quantities is compared
with recent observational trends that have been attributed to ozone depletion.
While in the literature these three quantities are assumed to have similar
variability, the results in this thesis show there are distinct differences
between them. In addition, comparison of the modeled natural variability with
the observed trends suggest that the observed trends in these three metrics are
not decisively outside of the natural variability.

Next, the relationship between oceanic heat and carbon content is examined in a
suite of coupled climate model simulations that use different parameterization
settings for mesoscale mixing. The different parameterizations result in different
multi-decadal variability, especially in the Weddell Sea where the characteristics
of deep convection are changed. While there are differences in the variability,
there is a robust anti-correlation between global heat and carbon content in all
simulations. Global carbon content variability is primarily driven by Southern
Ocean carbon variability. This contrasts with global heat content variability,
which is primarily driven by variability in the southern mid-latitudes and tropics.

Finally, we explore the relationship between age and oxygen in the North Atlantic and find that in
both observations and a model, the assumed negative linear relationship between
age and oxygen is not found both within and directly below the ventilated
thermocline at the end of Line W. While observations show a
decoupling of the biologically-driven age-oxygen relationship, our model analysis
indicates that this phenomenon is relatively localized to Line W due to the
combination of relatively weak horizontal gradients in age and oxygen resulting
reduced along-isopycnal variability and vertical heave acting on a depth offset
between age and oxygen extrema.

\par
  \vspace{.5in}
  \textbf{Faculty Advisor and First Reader:} Professor Darryn Waugh \\
  Earth and Planetary Science Department, Johns Hopkins University \\
  \textbf{Second Reader:} Professor Anand Gnanadesikan \\
  Earth and Planetary Science Department, Johns Hopkins University

\end{abstract}

%%% Local Variables:
%%% mode: latex
%%% TeX-master: "thesis"
%%% End:
