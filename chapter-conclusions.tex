\chapter{Conclusions}
\label{cha:conclusions}

\section{Summary of results}

The main findings of this thesis are summarised as follows:
%\begin{enumerate}

\paragraph{Application of moment diagnostics.} It has been demonstrated that
vortex moment diagnostics can be successfully applied to the geopotential height
field, giving similar results as when applied to conservative fields such as
PV. This provides a semi-Lagrangian (or vortex-centric) method which can be
readily used to describe the geometry of the stratospheric polar vortex in
climate model simulations.

It has been further shown that a simple threshold-based method can be applied to
the vortex moment diagnostics in order to identify split and displaced vortex
events. The majority of events identified in this way coincide with events
defined by other methods, and capture equally extreme vortex states.

\paragraph{The stratospheric polar vortex in climate models.} The first
multi-model comparison of stratospheric polar vortex geometry, including split
and displaced vortex events, has been carried out using the
stratosphere-resolving CMIP5 models. A wide range of biases have been identified
in the geometry of the stratospheric polar vortex among models. Some models have
a vortex which is on average too equatorward, others too poleward, while the
majority of models have a vortex which is too circularly symmetric. Models also
vary widely in their frequency of split and displaced vortex events. However,
the nature of these events is largely in agreement with observations, in
particular the fact that split vortex events appear more barotropic and
displaced vortex events are more baroclinic in nature.

\paragraph{Stratosphere-troposphere coupling in climate models and
  observations.}  In reanalysis data, using the geopotential height-based vortex
moments method, a stronger tropospheric NAM signal is seen following split
vortex events than displaced vortex events.

In the CMIP5 models, the tropospheric NAM signal following both split and
displaced vortex events is weak on average. There is no consistent difference
between the two apart from close to the onset of events when there is a negative
anomaly for split vortex events which extends barotropically through the depth
of the atmosphere. However, looking at two-dimensional tropospheric anomalies in
mean sea-level pressure following split and displaced vortex events shows some
consistent features. A negative NAO-like signal is seen which is of similar
magnitude following both types of event. The Pacific response is much less
robust, with some models simulating negative pressure anomalies, and others
positive. The discrepancy between the Atlantic and Pacific responses suggests
that the annular mode may not be a good metric for stratosphere-troposphere
coupling in the NH.

Almost all models show more negative sea-level pressure anomalies over Siberia
following displaced vortex events than split vortex events. Overall, the
differences in the surface signals following the two types of events are
approximately co-located with the difference in lower-stratospheric geopotential
height, which in turn follow stratospheric PV anomalies. A similar pattern is
also seen in tropopause height in reanalysis data.

\paragraph{Predictability of the polar stratosphere.} Using hindcast simulations
produced by a stratosphere-resolving seasonal forecast system, no skill has been
found in the prediction of NH SSWs or split or displaced vortex events at lead
times beyond one month.

On the other hand, skillful prediction of the SH stratospheric polar vortex
during the austral spring at seasonal lead times has been found. This skill is
greater than a persistence forecast; indeed, a strong late-summer polar vortex
is related to a weak spring vortex, indicating the importance of
preconditioning. Using the observed relationship between the strength of the
stratospheric polar vortex and polar ozone, it was possible to produce skillful
forecasts of interannual variations in polar stratospheric ozone depletion. This
prediction is at longer lead times than previous forecasts. Because interannual
variability is significant when compared to the long-term ozone depletion trend,
and has a significant impact on UV radiation reaching the Earth's surface, such
forecasts are likely to be of some interest to populations in the SH.

A further feature of the hindcast simulations is that the year 2002, in which
the only observed SH SSW occurred, is also the most extreme of the hindcasts
with almost all ensemble members simulating negative stratospheric wind
anomalies. It was also one of only 2 out of 210 ensemble members which simulate
SH SSW-like events (although these are displaced vortex events, rather than the
split that occurred). This suggests that an increased likelihood of the 2002
event may have been detectable almost two months in advance.

\paragraph{Stratospheric influence on tropospheric predictability.} The same
seasonal forecast system produces skillful forecasts of the austral spring mean
surface SAM at one month lead times. It also accurately simulates the surface
temperature pattern associated with the SAM, such that the SAM forecast skill
leads directly to skillful surface temperature forecasts over much of Antarctica,
New Zealand, and eastern Australia. Interestingly, these forecasts were found to
be more skillful during October--November (2 month lead time), than September (1
month lead time). The same pattern is replicated in a statistical hindcast which
takes as its only input the polar-cap mean geopotential height at 10~hPa on 1st
August. The pattern cannot, however, be replicated by a statistical forecast
based on the ENSO index. This suggests, therefore, that the tropospheric skill
during October-November is largely attributable to the influence of the
predictable stratosphere during this time. The October--November stratospheric
SAM is, in turn, highly predictable due to a strong negative correlation with
the 1st August stratospheric SAM. The fact that the stratospheric influence is
greatest in October-November is also backed-up by observational evidence which
shows the largest stratosphere-leading correlations with the surface during this
time. These results highlight the importance of including a well-resolved
stratosphere and accurate stratospheric initial conditions in seasonal forecast
systems.

%\end{enumerate}

\section{Limitations and further investigations}

The work presented in this thesis has raised a number of questions, and its
limitations have motivated future investigations. Some of these ideas are
discussed below:

\paragraph{What is required for a realistic stratosphere?} Several studies over
the past decade have demonstrated that a more realistic climate and improved
weather forecasts can be achieved using models which resolve the
stratosphere. This has proved persuasive to modelling centres, leading an ever
increasing number to include a representation of the stratosphere. Much of the
work in this thesis has reaffirmed and provided a more detailed picture of the
important role of the stratosphere in surface weather and climate. However, we
have also clearly seen that a high-top is not a sufficient condition for a
realistic stratosphere. A major challenge for the stratospheric community is to
identify where limited computing resources should be best spent in simulating
the stratosphere.





\paragraph{Synchronisation of the stratosphere and troposphere?} A large part of
this thesis has focussed on developing an increased understanding of the spatial
stucture of stratosphere-troposphere coupling. However, the mechanisms discussed
have retained the traditional temporal chain of causation of the form: \emph{A
  causes B; B causes C} etc.. In the real, chaotic atmosphere, it is unlikely
that such a simple mechanism exists. A new approach to understanding
stratosphere-troposphere coupling could focus on the synchronisation of modes of
variability. Indeed, we have seen here that such modes may be important because
of the barotropic nature of split vortex events, suggesting an excitation of the
barotropic mode during these events.



In principle, a similar technique can be applied to study
stratosphere-troposphere coupling. This could look, for instance, at whether
stratospheric and tropospheric modes are synchronised following particular
events, such as SSWs. A difficuly in this case is deciding which are the
relevant modes of variability. We could choose the NAM, although, as discussed
previously, this has different physical interpretations in the troposphere and
stratosphere.

The method decomposes a given time series
into a finite number of `modes', each of which have a characteristic
frequency. Unlike Fourier analysis, this frequency is allowed to vary to some
degree, so the modes need not be perfectly periodic. As such, it is more
applicable to time series of finite length and with a pronounced seasonal
variability, such as is seen in the atmosphere.



%%% Local Variables:
%%% mode: latex
%%% TeX-master: "thesis"
%%% End:
