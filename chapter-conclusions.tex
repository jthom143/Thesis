\chapter{Conclusions}
\label{cha:conclusions}

\section{Summary of results}

The focus of this thesis has been on quantifying and understanding the natural
variability of the atmosphere and ocean climate system. In particular, three features
were chosen for their importance: the Southern Hemisphere westerly jet, globally
integrated carbon and heat content, and North Atlantic age and oxygen concentrations.

In the Southern Hemisphere, a recent observed strengthening and shift in the westerly
jet in addition to a positive SAM index has prompted numerous research studies to
designate attribution to the observed trends. Very few studies, however, have
focused on the multi-decadal natural variability of the Southern Hemisphere
extropical circulation. In Chapter 2, through the analysis of CMIP5 pre-industrial control
simulations, we quantified and compared the \textit{modeled} natural variability
of multiple climate models and compared this natural variability with multiple
observational estimated trends in the SAM index, westerly jet location, and
westerly jet strength. We found that the observed trend in the SAM index is not
decisively outside the natural variability as simulated by the CMIP5 models.
On the other hand, the observed trend in the jet location and jet strength was
at the edge of the natural variability of the CMIP5 models. These results suggest
that while these three metrics are often assumed to be interchangeable, the observed
trends relative to the nodeled natural variability were all different, suggesting
this assumption is not strong.

In order to quantify the natural variability of oceanic carbon and heat content,
we used multiple climate simulations from a single, coarse-resolution, climate model.
In Chapter 3 we showed that in this model, deep-ocean convection in the Weddell Sea
drives a global response in both oceanic carbon and heat content, with the result
of a strong negative correlation between the two quantities. Using multiple simulations
with different parameter settings for mesoscale mixing, we varied the variability
of this Weddell Sea convection, to show that the negative relationship between oceanic
carbon and heat content was robust. We showed that this anti-correlation
between global oceanic carbon and heat is due to a difference in response in the
southern mid-latitudes and tropics.

Finally, in Chapter 4, we examined the relationship between age and AOU in the
North Atlantic ocean using observations from WHOI Line W and in a global climate model.
Our analysis suggests that the relationship between age and AOU, which is typically
assumed to be strongly positive, is more complicated than assumed. The
Pearson correlation coefficient between ocean age and AOU in the Line W observational
data is positive, with the exception of a region of near-zero correlation along
the 27.0 netural density surface and a region of negative correlation deeper in the
water column. Comparing the observational age-AOU relationship with the climate model,
we see a similar region of near-zero correlation within the thermocline along
Line W. Our model analysis suggests that this region of reduced-positive correlation
is due to a combination of limited along-isopycnal variability, along with a
offset between the age and AOU maximum depths.





%\begin{enumerate}

\section{Limitations and further investigations}

The majority of the work presented in this thesis has focused on the analysis of
global climate models (GCMs). This is in part because GCMs are a powerful and useful
tool for understanding the climate system, and also because of the relatively poor
temporal resolution and spatial extend of observational data. As more observational
data is collected, in part due to the recently deployed SOCCOM project in the Southern Ocean along
with advances in technology, it would be interesting and insigntful to supplement
this analysis with further observational analysis.

This analysis has also prompted multiple additional lines of research questions.
First, it would be worthwile to extend the analysis presented in Chapter 2 to other
Southern Ocean quantities that have experienced significant trends in recent decades, such
as Southern Ocean SST. Additionally, with the next generation of CMIP models, and
an additional decade of observational data, the analysis could be repeated to see
if the conclusions change.

With regard to Chapter 3, only one model was included in the analysis. Given the
documented strong influence of Weddell Sea convection on global climate in this model,
tt would be worthwile to repeat the analysis with a different climate model.
Using a climate model from a different modeling center to examine the natural variability
of globally integrated oceanic carbon and heat would provide important insignt to
wether or not the negative correlation is robust.

Chapter 4 also proved surprising in that the expected relationship was more complicated
than assumed. An additional line of research could include looking at additional
observational data to see if similar regions of near-zero age-AOU correlations
are seen near the center of the gyre circulation. This analysis could prove critical
for subsequent analysis which depend on the age-AOU relationship, such as estimating
the OUR and other biological activitiy measures.





%%% Local Variables:
%%% mode: latex
%%% TeX-master: "thesis"
%%% End:
