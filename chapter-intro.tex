\graphicspath{{figures/chapter-intro/}} % Specifies the directory where pictures are stored


\chapter{Introduction}
\label{cha:intro}


Anthropogenic climate change has been a hot topic amongst scientists and the public
for the past decade. A few of the primary questions associated with climate change
relate to how scientists know that climate change is caused by humans and where
and when the effects of climate change will be seen. Both of these questions require
an understanding about what causes the climate to change over time. Earth's climate
is a complex dynamic system and has changed constantly throughout Earth's history.
Being able to separate the climate's natural fluctuations from the change caused by
anthropogenic activity is essential to being able to accurately predict and detect
future changes in climate.

\section{Natural Variability}

One of the primary challenges of assessing the impact of anthropogenic influence
on Earth's climate is a lack of understanding of the natural fluctuations of the
climate system. Internal climate variability is simply how Earth's climate varies
in time without any external forcing. External forcing can be natural in source,
including volcanic eruptions, and solar forcing, or anthropogenic, including
human emissions of greenhouse gases and human caused land use changes. The term
natural variability on the other hand is typically used to refer to fluctuations
in Earth's climate caused solely by natural forcing (internal or external), and
not including anthropogenic influence.

Natural variations in the climate system, also referred to as ``climatic noise''
\citep{Madden1976a,Schneider1994,Wunsch1999,Feldstein2000},
are a result of non-linear processes in the atmosphere
and ocean operating on a variety of spatial and temporal scales. The most common
example of natural variability is the El Ni\~no-Southern Oscillation (ENSO),
which is a result from the non-linear interactions between the atmosphere and
ocean in the Equatorial Pacific. While ENSO is perhaps the most famous of all the
atmospheric climatic modes of variability, there are multiple modes of variability
in the atmosphere and ocean which interact with each other.

A further complicating matter is that climate variations happen on all timescales:
paleo-climatic, centennial, multi-decadal, decadal, inter-annual, and seasonal, in
addition to on all spatial scales from regional to global. Global variability on
long timescales is generally the best understood, however regional variability is
generally the most important when it comes to understanding and interpreting recent
trends in observational data.

Quantification of natural variability is important to climate science for many
reasons. First, natural variability is key to determine climate predictability.
Climate predictability was first defined in 1975~\citep{TheNationalAcademyofScience1975}
as a measure of signal-to-noise, where the signal is any potentially predictable
long-term (> 1 year) climate feature, and the noise is natural variability. Having
an estimate of natural variability allows us to answer the question - how large
does a trend need to be in order to detect it? Without an idea of what the natural
variability of a given feature is, it is nearly impossible to determine if a
signal lies within the noise of the climate system.

Another reason quantifying natural variability is important relates to projection
uncertainty. Characterizing uncertainty for climate change projections is important
for purposes of detection and attribution and for strategic approaches to adaptation
and mitigation~\citep{Deser2012}. Uncertainty in future climate change comes from
three main sources: forcing, climate model response, and internal variability
\citep{Lovenduski2015,Hawkins2009,Tebaldi2007}.

A final example of where understanding natural variability is important is in the
evaluation of global climate models. Because of observational limitations and the
inability to experiment on the entire earth system, climate models are developed
and used to further our knowledge of the planet and climate system. Climate models
are typically validated against the mean state of a given climate feature. For
example, to assess the accuracy of the ocean module in a climate model, one might
compare the average sea surface temperature (SST) in a pre-industrial control model
run to the average of many years of SST observations from similar period (via proxy
records). While this method is useful for assessing the mean state of the climate
model, it does not give any information about the variability of the model. The
model could have vastly larger variance than the observations, which would not be
captured by comparing the means.

One of the largest challenges with estimating natural variability is a lack of
observational estimates. In an ideal situation, scientists would be able to quantify
natural variability using observations of the Earth system. However, because natural
variability operates on all time and spatial scales, this would require millions
of observations across the entire globe. Global natural variability over thousands
of years is estimated using proxy records of Earth's climates; however, regional
variability on shorter timescales (decadal to multi-decadal) is much more of a
challenge to quantify because it requires finer resolution of observations.

To supplement the limited observational estimates of natural variability, scientists
often turn to global climate models to quantify natural variability. Unlike
observational records, global climate models have a consistent spatial and temporal
resolution. Global climate models can also be used to simulate multiple realizations
of Earth's climate over thousands of years, to generate a statistical distribution
of climate variability in absence of external forcing. Because no single climate
model is a perfect representation of Earth's dynamical and biological processes,
multiple models are often employed to reduce the uncertainty due to model configuration.
This method of quantifying natural variability also comes with challenges. Primarily,
global climate models are very computationally expensive and require extensive
resources and time to generate a simulation. Additionally, to best quantify
natural variability, multiple simulations without any anthropogenic forcing should
be performed. Because of the emphasis on simulating future change, fewer resources
have traditionally been put toward conducting non-anthropognically forced simulations.

In this thesis, natural variability in both physical fields (winds and temperature)
and biogeochemically active fields (carbon and oxygen) is examined across multiple scales.


\section{Thesis Overview}
This thesis takes an investigative look at the multi-decadal natural variability
of three important components of the climate system: the Southern Hemisphere
westerly jet, oceanic carbon and heat content, and North Atlantic oxygen and age.
These three components of the atmosphere and ocean variability were selected
because of the important role they have in the response of the global climate system.

\subsection{Chapter 2 - Southern Hemisphere Westerly Jet}
The Southern Hemisphere (SH) westerly jet is incredibly important for driving the
Southern Ocean circulation, which has strong influence on global climate. The
strong, eastward flowing winds of the drive subsurface Ekman transport towards
the north. This northward flowing water is colder than the water it encounters
(and therefore denser) and subsequently sinks into the interior to form Antarctic
Intermediate Water - a process known as ventilation. Closer to the Antarctic
continent, overlaying easterly winds drive the surface water towards the south
where waters interact with the Antarctic sea ice and dense water is formed
(formation of Antarctic Bottom Water). The resulting divergence of water at the
surface of the Southern Ocean allows for deep water (primarily North Atlantic
Deep Water) to rise to the surface and interact with the atmosphere
(Figure~\ref{fig:carbon_cycle}).

This process of water rising to the surface, interacting with the atmosphere,
and subsequently sinking back into the interior allows for increased
atmosphere-ocean gas exchange and results in the Southern Ocean being very
influential on global climate. the Southern Ocean is one of the most important
oceans for regulating global climate. Recent estimates suggest that the Southern
Ocean is responsible for 75\% of the heat uptake and 30\% of the carbon uptake
\citep{Frolicher2015}.

Recent studies suggest that anthropogenic ozone depletion and greenhouse gas
induced warming has already had an impact on the SH westerly jet through a strengthening
and shift towards the Antarctic continent. In Chapter 2, we utilize 14
state-of-the-art global climate models from modeling centers around the world to
quantify the natural variability in the SH westerly jet. We then compare the
model-estimated natural variability to the recently observed trends in the jet
strengthening and pole-ward shift. Our results suggest that a combination of
natural variability and anthropogenic forcing are required to explain the observed
trends in the SH westerly jet.

\subsection{Chapter 3 - Oceanic Heat and Carbon Variability}
Chapter 3 turns to the role of natural variability in the Southern Ocean on the
global and regional oceanic heat and carbon budget. One particular phenomenon
that holds significant implications for global climate is deep ocean convection.
The high-latitude Southern Ocean is weakly stratified with cold fresh water
overlaying slightly warmer saltier water of nearly identical density
\citep{Martinson1991}. Deep convection in this region occurs when this weakly
stratified surface layer is perturbed and becomes denser than the under-laying
waters resulting in the entire water column turning over. This results in the
surface waters being subducted deep into the ocean, and old deep water being
brought up to the surface.

While only one of these deep convective events have been observed in the real
Earth's Southern Ocean (the Weddell Polynya which persisted through the winters
of 1974-1976) many global climate models have consistent periodic convection
events in this region~\citep{DeLavergne2014a}. Because of the intense exchange
of surface and deep waters associated with deep convection, these events have
been shown to have profound impacts on global climate~\citep{Gordon1982,Cabre}.

In Chapter 3, we use multiple simulations from a coarse-resolution climate model
to explore the temporal variability of oceanic carbon and heat and investigate
how this global heat and carbon variability is impacted by Southern Ocean deep
convective events. We show that the Southern Ocean deep convection has significant
influence on the magnitude of global oceanic heat and carbon variability, however
the relationship between oceanic heat and carbon is consistently anticorrelated
across model simulations. We demonstrate that this results from differences in how
the solubility and biological pumps of carbon respond to convective variability.

\subsection{Chapter 4 - Age and Oxygen Relationship}
Chapter 4 builds on this idea that biological and physical responses to changes
in circulation may differ. Oceanic age refers to the time since the ocean water
was last in contact with the atmosphere. This derived tracer is used to estimate
the rate of ocean ventilation and overturning circulation. Oxygen is another
tracer that is important to understanding ocean circulation and biological activity.
Age and oxygen are generally thought to have a strong negative correlation because
biological utilization reduces oxygen concentrations in the ocean interior where
the age is the oldest. This presumed relationship is often times used in ocean
biogeochemistry and oceanography to estimate changes in biological activity and
ocean circulation.

We focus on an observational data set in the North Atlantic and look at how
changes in oxygen and oxygen utilization relate to changes in ventilation age.
We show that in the observational record and in a global climate model simulation,
along Line W in the North Atlantic this expected relationship between age and oxygen
is more complicated due to the different spatial structure of sources of apparent
oxygen utilization and age in the deep ocean.


\begin{figure}
\centering
\includegraphics[width=33pc]{Southern_Ocean.jpg}
\caption{Schematic depicting Southern Ocean circulation. Figure by Ilissa Ocko, courtesy of Princeton University.}
\label{fig:carbon_cycle}
\end{figure}


%%% Local Variables:
%%% mode: latex
%%% TeX-master: "thesis"
%%% End:
