\chapter{Introduction}
\label{cha:intro}

\section{Natural Variability}

\subsection{What is Natural Variability?}
One of the primary challenges of assessing the impact of anthropogenic influence on Earth's climate is a lack of understanding of the natural fluctuations of the climate system. Internal climate variability is simply how Earth's climate varies in time without any external forcing. External forcing can be natural in source, including volcanic eruptions, and solar forcing, or anthropogenic, including human emissions of greenhouse gases and human caused land use changes. The term natural variability on the other hand is typically used to refer to fluctuations in Earth's climate caused solely by natural forcing (internal or external), and not including anthropogenic influence.

Natural variations in the climate system, also referred to as 'climatic noise', (Madden 1976; Schneider and Kinter 1994; Wunsch 1999; Feldstein 2000) are a result of non-linear dynamical and biological processes in the atmosphere and ocean operating on a variety of spatial and temporal scales. The most common example of natural variability is the El Nino-Southern Oscillation (ENSO), which is a result from the non-linear interactions between the atmosphere and ocean in the Equatorial Pacific. ENSO in turn impacts temperatures across much of the Northern Hemisphere, further interacting with other modes of variability.

A further complicating matter is that climate variations happen on all timescales: paleoclimatic, centennial, multi-decadal, decadal, interannual, seasonal, and on all spatial scales from regional to global.


\subsection{Why is natural variability important? }

Quantification of natural variability is important to climate science for many reasons. First, natural variability is key to determine climate predictability. Climate predictability was first defined in 1975 (Academy of sciences publication) as a measure of signal-to-noise, where the signal is any potentially predictable long-term (> 1 year) climate feature, and the noise is natural variability. (Answering the question - how large does a trend need to be in order to detect it?)
Projection uncertainty.

Another example of where understanding natural variability is important is in the evaluation of global climate models. Because of observational limitations and the inability to experiment on the entire earth system, climate models are developed and used to further our knowledge of the planet and climate system. Climate models are typically validated against the mean state of a given climate feature. For example, to assess the accuracy of the ocean module in a climate model, one might compare the average sea surface temperature (SST) in a pre-industrial control model run to the average of many years of SST observations from similar period (via proxy records). While this method is useful for assessing the mean state of the climate model, it does not give any information about the variability of the model. The model could have vastly larger variance than the observations, which would not be captured in comparing the means.

\subsection{Challenges quantifying natural variability?}


\section{Background}
As the largest reservoir for carbon and heat in the Earth system on decadal-to-centennial timescales (Figure 1.1), the ocean has a great amount of influence on global climate variability (references…).

\subsection{Surface Mixed Layer}
The atmosphere and ocean primarily interact through gas exchange in the surface mixed layer of the ocean. This top layer of the ocean is mixed and stirred by the overlying atmospheric winds. This mechanical forcing acts to create a layer that is nearly uniform with respect to the vertical tracer gradient. Gas exchange between the atmosphere and ocean is determined by atmosphere-ocean disequilibrium. For oceanic heat content, it is the disequilibrium between the atmosphere temperature and mixed layer ocean temperature that determines if heat is transfer between the two mediums. For carbon dioxide, it is the partial pressure of CO2, pCO2, that drives the gas exchange between the atmosphere and ocean.

Tracer concentrations in the surfaced mixed layer are balanced by this atmosphere-ocean gas exchange at the surface, lateral and vertical advection of the tracers, and tracer diffusion (ref). Lateral advection is primarily driven by the winds at the surface whereas vertical advection typically acts through entrainment at the base of the mixed-layer. Entrainment occurs when the depth of the mixed layer varies with time (typically the seasonal cycle). While diffusive fluxes in the surface mixed layer are usually much smaller compared to the other terms (Levy et al., 2013).

\subsection{Ocean Circulation }

The circulation of the surface ocean is primarily driven by the meridional gradient in the zonal surface winds. Because the zonal winds vary with latitude, the winds impart local vorticity (i.e. spin) into the ocean. This momentum, combined with the Earth's rotation (planetary vorticity, i.e. Coriolis force), sets up circular flow pattern in the ocean basins with anti-cyclonic flow in the subtropical-gyres (counter-clockwise in the NH) and cyclonic flow in the tropical and polar gyres (clockwise in the NH) with intensification on the westward side of the ocean basin.  This balance between the vorticity imparted by the wind-stress and the planetary vorticity is referred to as Sverdrup's balance (Sverdrup 1947).
Ekman transport (Ekman 1905) results in net transport at a right angle to the surface current (in Northern Hemisphere) resulting in water converging at the center of the sub-tropical basin gyres.

Southern Ocean is unique because it is the only area in the global Oceans where the ocean extends around the entire globe. Additionally, this ocean is aligned with the over-lying sub-tropical westerly jet. The result is a surface current which wraps around the entire globe, the Antarctic Circumpolar Current, with an average strength of 130 Sv.

Via Ekman pumping, the westerly winds over the Southern Ocean transport water northwards that forces water from depth up to the surface, causing shoaling of isopycnals (density surfaces). Due to the northward volume transport of water, the Southern Ocean forms a key area in the large-scale overturning circulation in the ocean.

While it's the interaction of the oceanic surface waters and the atmosphere that drive the air-ocean exchange of tracers, it's the transport of these tracers into the deep ocean that allows the ocean to function as such a substantial sink for atmospheric quantities (most notably carbon and heat). This exchange of surface water into the deeper ocean, requires a substantial change in density (densification), while the exchange of deep water to the surface typically requires strong mechanical forcing (i.e. wind forcing).

Because of these requirements, this surface-deep water exchange localized to only a few places in the global ocean - usually in the high latitudes.

Southern Ocean Circulation

\subsection{High latitude atmospheric circulation and the impact on the ocean. }
\subsection{Ocean Biogeochemistry}

This chapter gives an idea of how complex and interrelated the atmosphere and ocean climate system is, and why understanding natural climate oscillations is so important. In the next section, the detailed aims of this thesis are discussed, and the structure is outlined.



\section{Overview and Aims}
This thesis takes an investigative look at the multi-decadal natural variability of three important components of the climate system: the SH westerly jet, oceanic carbon and heat content, and North Atlantic oxygen and age. These three components of the atmosphere and ocean variability were selected because of the important role they have in the response of the global climate system.

In Chapter 2, multiple state-of-the-art global climate models are examined in order to quantify the natural variability of the SH westerly jet. This natural variability is then compared with recent observational trends that have been attributed to anthropogenic activities. This analysis provides insight into the climatic predictability of the SH westerly jet.

Another incredibly important feature of climate variability is the variability of oceanic carbon and heat content. In Chapter 3, a single Earth System Model is deployed to investigate the variability of both oceanic carbon and heat content and the relationship between the two quantities.

In Chapter 4,

Finally, conclusions in Chapter 5.


%%% Local Variables:
%%% mode: latex
%%% TeX-master: "thesis"
%%% End:
