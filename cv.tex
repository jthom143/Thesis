\begin{cv}
\addcontentsline{toc}{chapter}{Curriculum Vitae}

  \begin{centering}
    \Huge{Jordan Thomas}\\
    \normalsize{Born: May 4, 1990}\\
    \normalsize{Wichita, KS}\\
  \end{centering}
  \vspace{1cm}
  \noindent\Large{Education}\\
  \vspace{0.2cm}
  \noindent\rule{\textwidth}{0.4pt}
  \normalsize{
  \noindent \textbf{Johns Hopkins University} (August 2012 -- Present) \hfill Baltimore, MD\\
	\indent PhD in Earth and Planetary Science\\
	\indent Masters in Earth and Planetary Science\\
	\indent Dissertation: Investigating Natural Variability in the Climate System\\
  \noindent \textbf{Pennsylvania State University} (August 2008-- May 2012) \hfill University Park, PA\\
	\indent Bachelors of Science in Meteorology\\
	\indent Concentration in Atmospheric Sciences \\
  \noindent \textbf{University of Southampton} (January 2011 -- July 2011)\hfill Southampton, UK\\
	\indent Minor in Oceanography\\
  }

  \vspace{1cm}
  \noindent\Large{Publications}\\
  \vspace{0.2cm}
  \noindent\rule{\textwidth}{0.4pt}
  \normalsize{
  Thomas J. L., Waugh D. W., and Gnanadesikan A. (2018) ``Relationship between ocean heat\\
  \indent and carbon variability''. \textit{Journal of Climate}. 31. 1467--1482. \\
  Brune W. H., Baier B. C., Thomas J.L., Ren X., Cohen R. C., Pusede S. E., Browne E. C.,\\
  \indent Goldstein A. H., Gentner D. R., Keutsch F. N., Thornton J.A., Harrold S., Lopez-\\
  \indent Hilfiker F. D. and  Wennbergm P. O. (2016) ``Ozone production chemistry in the pres-\\
  \indent ence of urban plumes''. \textit{Faraday Discussions}. 189. 169--189. \\
  Thomas J. L., Waugh D. W., and Gnanadesikan A. (2015) ``Southern Hemisphere extra- \\
  \indent tropical circulation: Recent trends and natural variability''.
  \textit{Geophysical Research Let-}\\
  \indent \textit{ters}. 42. 5508--5515. \\
  Pusede S. E., Gentner D. R., Wooldridge P. J., Browne E. C., Rollins A. W., Min K.-E., \\
  \indent Russell A. R., Thomas J. L., Zhang L., Brune W. H., Henry S. B., DiGangi J. P., Keutsch \\
  \indent F. N., Harrold S. A., Thornton  J. A., Beaver M. R., St. Clair J. M., Wennberg P. O., \\
  \indent  Sanders J., Ren X., VandenBoer T. C., Markovic M. Z., Guha A., Weber R., Goldstein  \\
  \indent A. H., and Cohen R. C. (2014) ``On the temperature dependence of organic reactivity,\\
  \indent nitrogen oxides, ozone production, and the impact of emission controls in San \\
  \indent Joaquin Valley, California''. \textit{Atmospheric Chemistry and Physics}. 14. 3373--3395.

  }
  \clearpage
  \thispagestyle{myplain}
  \vspace{1cm}
  \noindent\Large{Experience}\\
  \vspace{0.2cm}
  \noindent\rule{\textwidth}{0.4pt}
  \normalsize{
  \textbf{Johns Hopkins University | RESEARCH ASSISTANT}\\
  September 2012 -- Present
  \begin{itemize}
   \item Researched the impact of ozone depletion on atmospheric and oceanic circulation biogeochemistry.
   \item Performed statistical analysis on output from 27 coupled climate model simulations (CMIP5) using Python to determine that recent trends in atmospheric quantities are likely caused by anthropogenic activities.
   \item Designed and executed model simulations using a fully coupled General Circulation Model to investigate relationship between ocean heat and carbon content.
   \item Analyzed model output using post-processing techniques in Python and MATLAB, focusing on atmosphere-ocean interactions, ocean dynamics, ocean biogeochemistry and ocean acidification.
   \item Presented dissertation research at multiple scientific conferences and invited talks including at University of Pennsylvania and MIT.
  \end{itemize}
  \textbf{Johns Hopkins University | TEACHING ASSISTANT}\\
  September 2015 -- December 2015
  \begin{itemize}
    \item Developed and taught interactive lessons focused on global and environmental change.
    \item Facilitated bi-weekly review sessions and held weekly office hours to assist with questions and homework.
  \end{itemize}
\textbf{Pennsylvania State University | UNDERGRADUATE RESEARCH ASSISTANT}\\
August 2009 -- May 2012
\begin{itemize}
  \item One of two students selected from Penn State to participate in CalNEX 2010, an air quality research campaign in Bakersfield, CA.
  \item Studied oxidation photochemistry and preformed analysis and model comparison with collected data.
  \item Developed an instrument to measure in-situ tropospheric ozone production.
  \item Analyzed air-quality data to diagnose large scale meteorological patters.
\end{itemize}


  }

\end{cv}
